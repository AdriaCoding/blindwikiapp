\section{Introducción}

BlindWiki es una red de audio geolocalizada que permite a personas con discapacidad visual total o parcial compartir sus experiencias mediante grabaciones sonoras utilizando teléfonos inteligentes. Creada en 2014 por Antoni Abad, la plataforma no se limita a documentar dificultades y barreras urbanas, sino que constituye un repositorio de experiencias, opiniones e historias que genera una cartografía colaborativa y creativa de lo invisible.

El proyecto tuvo su origen como una iniciativa para dar voz a las personas con discapacidad visual, permitiéndoles documentar y compartir su percepción única del entorno urbano. Desde su creación, BlindWiki ha experimentado una notable expansión internacional, desarrollándose en ciudades como Roma (2014-2015), Sydney (2015), Berlín y Wrocław (2016), Venecia (2017), Valencia (2020) y São Paulo (2022), adaptándose a cada contexto cultural y lingüístico.

La aplicación móvil de BlindWiki, disponible tanto para Android como iOS, permite a los participantes grabar audio específico del lugar y publicarlo inmediatamente en la plataforma. Los usuarios pueden desplazarse por sus ciudades mientras publican y reciben descripciones de audio geolocalizadas, historias, obstáculos o crónicas previamente contribuidas a través de la app. Esta funcionalidad facilita la creación de un mapa sensorial colectivo que enriquece la experiencia de navegación urbana para personas con discapacidad visual.

\subsection{Motivación y Objetivos}

La necesidad de rediseñar la aplicación surge de los avances tecnológicos en los dispositivos móviles y sus sistemas operativos, así como de la comunidad internacional de usuarios que demanda mejoras en la accesibilidad y funcionalidad.

El stack tecnológico de la aplicación original se compone de PhoneGap/Cordova como framework base para el empaquetado, e Ionic 1 con Angular.js para la interfaz de usuario y lógica de la aplicación. Esta arquitectura, aunque permitió un desarrollo multiplataforma eficiente en su momento, ha quedado obsoleta frente a los estándares actuales de desarrollo móvil, lo que dificulta la implementación de nuevas funcionalidades y afecta al rendimiento en dispositivos modernos.

La expansión internacional del proyecto ha levantado también otra problemática: las barreras lingüísticas entre los usuarios limitan enormente sus interacciones. Con el fin de romper estas barreras, surge la motivación de implementar módulos de transcripción y traduccion automática del habla. El objetivo es que cualquier usuario pueda ser capaz de obtener traducciones de las notas de voz de otros usuarios, en cualquiera de los 14 idiomas que soporta BlindWiki \ref{tab:supported-languages}:

\begin{table}[ht]
    \centering
    \caption{Idiomas soportados en BlindWiki}
    \label{tab:supported-languages}
    \begin{tabular}{|l|l|l|}
        \hline
        \textbf{Idioma} & \textbf{Código} & \textbf{Estado} \\
        \hline
        Catalán & ca & Activo \\
        Alemán & de & Activo \\
        Inglés & en & Activo \\
        Español & es & Activo \\
        Francés & fr & Inactivo \\
        Gallego & gl & Activo \\
        Húngaro & hu & Inactivo \\
        Italiano & it & Activo \\
        Japonés & ja & Inactivo \\
        Coreano & ko & Activo \\
        Polaco  & pl & Activo \\
        Portugués & pt & Activo \\
        Chino & zh & Inactivo \\
        Árabe & ar & Inactivo \\
        \hline
    \end{tabular}
\end{table}

Entre el conjunto de metadatos que acompaña cada mensaje de voz, está una lista de \textit{tags} que los usuarios introducen voluntariamente. Estos tags suelen tener relacion con conceptos del entorno urbano. Algunos ejemplos frecuentes son \textit{restaurant}, \textit{danger}, o \textit{iglesia}. Para muchos usuarios, tener que dedicar tiempo a escribir manualmente los tags puede resultar pesado. Por ese motivo, en la app muchas notas de voz quedan sin etiquetas, y probablemente desapercibidas por los demás usuarios.
El tercer objetivo consiste en desarrollar un módulo capaz de generar etiquetas para las notas de voz de forma automática. 

